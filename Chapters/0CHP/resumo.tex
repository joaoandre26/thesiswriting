\TitlePage
  \vspace*{55mm}
  \TEXT{\textbf{Palavras-Chave}}
       {Gases de petróleo liquefeito, FFT, Acelerómetro, Piezoeléctrico}
  \TEXT{\textbf{Resumo}}
       {Esta dissertação apresenta um estudo efectuado com o intuito de desenvolver um dispositivo capaz de medir a quantidade de gás líquido dentro de uma garrafa, de forma não evasiva. A investigação realizada resultou na identificação de diferentes métodos de medição de nível, como por exemplo a medição do peso, a análise das vibrações resultantes da estimulação externa, entre outros. Da investigação para a escolha do método de identificação de nível de líquido, optou-se por desenvolver o trabalho com base na análise das vibrações produzidas quando estimulada a garrafa.}
  \TEXT{}
       {O trabalho dividiu-se em diferentes objetivos, todos com vista ao desenvolvimento de um dispositivo que se enquadre no propósito. Inicialmente identificou-se a resposta da garrafa a um estimulo externo, ao analisar a sua resposta em frequência, para tal recorreu-se a um microfone para captar a mesma. De seguida procedeu-se à implementação e validação da FFT em um microcontrolador de pouco consumo energético, com o intuito de verificar a viabilidade de fazer a análise em frequência no microcontrolador. Após isto o trabalho centrou-se na seleção e testes a sensores, para substituir o microfone utilizado inicialmente, para isso foi utilizado um acelerómetro e um piezoeléctrico. Por fim procedeu-se ao desenvolvimento de um algoritmo de identificação de nível, com base nas vibrações captadas pelos sensores, que permita distinguir as diferentes garrafas utilizadas em testes.} 
\EndTitlePage
\titlepage\ \endtitlepage % empty page
%%English version
\TitlePage
  \vspace*{55mm}
  \TEXT{\textbf{Keywords}}
       {Liquefied petroleum gas, FFT, Accelerometer, Piezoelectric}
  \TEXT{\textbf{Abstract}}
       {This dissertation presentes the study made with the intent of developing a device capable of measuring the amount of liquid gas inside a bottle, in a non evasive way. The realized investigation resulted in the identification of diferent methods to measure the level, as for example the weight measure, the vibrations analisys from the external stimulation, among others. From the investigation to select the liquid level identification, it was chosen to develop the work based on the analisys of the vibrations produced when stimulated.}
  \TEXT{}
       {The work was divided with different objectives, all of them with the development of a device that fits in the purpose in perspective. Initially was identify the response of the bottle to a external stimulation, by the analisys of the frequency response, for that a microphone was used to capture the response. This was proceeded by the implementation and validation of the FFT in a low poer microcontroller, with the intent to verify the viability of the frequency analisys in the microcontroller. After this, the work focused in the selection and the test of sensors, to substitute the microphone inicially used, for that a accelerometer and a piezoeletric was chosen. Finally, was developed an algorithm to the level identification, based on the vibrations captured with the sensors, that allow to distinguish between the three bottles used in the tests.}
\EndTitlePage
\titlepage\ \endtitlepage % empty page

\cleardoublepage
\chapter{Conclusions \& Future Work}\label{chap:finalremarks}
% To do: 
% Color Scheme
% Red - Not started
% Yellow - In progress
% Blue - Done, under approval
% Green - Done and approved

%\todo[inline,color=blue!40]{closed - v1.2 finished 24/07/2021, to review}
%\todo[inline,color=red!40]{*1 - Conclusions}\label{sec:Conclusions}
%\todo[inline,color=red!40]{*2 - Future Work}\label{sec:FutureWork}
\section{Conclusions}
The objective of this work was to develop a device capable of measuring the amount of liquid gas inside a \acrshort{lpg} bottle, without using an invasive technique and suitable to be used in the bottles that  are already available in the market. Beside the device for this purpose, the final solution should also be able to transmit to the distributor the measured value, so he would be able to properly plan a distribution route. That would imply developing a device, having local processing and communication resources, able to report measurements and be integrated in a \acrshort{iot} framework.

Considering the practical objective of this work, was firstly performed a background study, in order to analyze what type of solutions were already developed with a similar purpose, before proposing an architecture for the system. Taking into account that the main focus is in the development of the device to measure the liquid level, the background study was oriented to the research of techniques to perform the measurement of the liquid level and thus selecting the proper components to develop the work.

The starting point was to find information on how to acquire data relative to the liquid level, i.e. what type of techniques can be used for that purpose. A couple of methods were presented, each one of them, returning different types of information. The working principles range from measuring the weight, the vibration or using ultrasound. In the end the choice fell to the measure of the vibrations, mainly due the fact that this could result in a smaller and practical device. 

The sensor requires the capability of measuring small vibrations produced when stimulating the system. For that were selected two sensors capable of measuring the vibrations. From the same background study it was found that there is a relation between the liquid level and the amplitude and center of the resonance frequencies, thus being necessary to process the data acquired to carry out a spectral analysis, being used for that purpose an \acrshort{fft}.

Taking into consideration the requirements of the system, was proposed an architecture, to acquire, process and return the liquid level. This proposal considers only a potential final product. Since it was not possible to know beforehand the effectiveness of the methods, the work was divided in several smaller steps, with the purpose of creating the foundations for in the future build the proposed architecture.

The steps consist in an initial study of the response of the system to a stimulation, to verify if the approach is feasible and if there is in fact a relation between the liquid level and the frequency of the vibration, using \acrshort{matlab}. Then it was carried out the implementation of the \acrshort{fft} in a microcontroller device to assess the feasibility of the approach. After it, testing the sensors in order to find if they are suited for the application and to determine if the same relation level vs frequency is found. And finally the development of an identification method of the liquid level, based on the conjugation of the last two steps. If there are positive results from all these steps, a setup can be develop according to the proposed architecture.

The tests performed in the different stages of the practical part of this work revealed different and distinct results. The first two steps, although with different purposes, were the ones that revealed the best results. The first, that was intended to study the response of the system to a stimulation, resulting in the confirmation of the first hypothesis, show that the lower the liquid level the higher is the frequency produced in the vibration and the inverse for the maximum liquid level in the bottle. This test was also important in the determination of the variables to consider in the analysis in frequency, that were used in the tests that followed, mainly in the last two steps.

The second test, with the purpose to evaluate the accuracy of the implementation of the \acrshort{fft}, result in precise results for signals randomly generated, allowing to conclude that the processing of the signals captured with the sensors and processed in the microcontroller would also return viable data to determine the liquid level.

The third test, that focused on the study of the resulting data from the sensors, was the test that revealed the most disappointing results. Although there were positive results in the test that allowed the execution of the last step, these tests had different variables to consider and to control. The variable that revealed to be more difficult to control, in a potential application, was the stimulation. It was not possible to obtain meaningful results by stimulating the system with the solenoid. The only viable results were obtained by the stimulation with an hammer, which is not consistent enough nor practical, but was enough to the execution of the last step. Another variable that was difficult to control was related to the coupling of the sensors. By using the load strap and the magnet to mount the accelerometer, we were able to obtain acceptable results, for the mentioned tests. The other options did not reveal any useful results. This was the most difficult variable to control, even though there are some documents related to this topic, none of them had a similar application. 

The last step was to evaluate if it was possible, considering the results of the processed data, to develop an algorithm capable of determining the liquid level. In this test, the results obtained for the identification algorithm were not as good as expected, but it was possible to determine it considering the first frequency peaks identified, as mentioned in the results analysis. Although there were acceptable results, this is also a step to improve.

In a final overview of all the work developed and the results obtained, it is possible to conclude that there are several aspects that influence the results, most of them already explored. There are other aspects that do not influence directly in the results, but in development itself. Considering all of this, it is important to mention that, even though the final purpose of the work was not complete, due to several constraints, it was possible to obtain positive results. These results can be used in a future work as a good starting point to the continuation in the development of the final solution, they reveal what can be explored and what can be discarded in a future. From the developed work and the problems faced along it, various suggestions emerged that will be explored in the following section. 
\section{Future Work}\label{sec:FutureWork}
The work developed in the context of this dissertation, although allowed the formation of a foundation to the development of a device with the capabilities of measuring the liquid level, as intended, still has several aspects that can be improved and others that need to have a different approach to fully attain the objectives. In order to improve the already developed work and having in mind all the problems faced along the development of it, there will be presented some suggestions to improve the work developed so far. 

% Stimulation method
The first suggestion is concerns the stimulation of the system, from the results is visible that hitting with a hammer return results, but is not practical, so in this case the suggestion would be to use also use a solenoid, capable to create a impact as strong as with an hammer.

The second suggestion, also concerns to the stimulation of the system, but is completely different to the one used, consisting in the use of piezoelectric sensors, emitting a sine wave with a certain frequency to stimulate and capture the response with another piezoelectric. By varying the frequency and correlating the emitted and received signals, is expected to determine the liquid level. This method was explored in chapter~\ref{chap:stArt} in the work from~\citeauthor{jahnLevelSensorFluids2014a}\cite{jahnLevelSensorFluids2014a}. As mentioned, is a completely different approach, but in their study proved to have good results.   
% Coupling 

One of the problems faced along the work is related with the coupling of the sensor with the bottle. Although some good results were obtained with the magnet glued to the accelerometer, the best results were obtained with the load strap, in a future application the use of a load strap end up not being a very practical solution. The suggestion in this case is to use a stronger neodymium magnet glued to the accelerometer.

In the hardware and the software developed there was always a problem related with the noise at the output of the sensors . For this case there is a suggestion that is widely used in signal processing for this type of situation and has a simple implementation, which is the use of moving average filters, this type of filters are optimal to the reduction of random white noise, while keeping the response. We will not get into details about these filters, for more information about the use of this filter~\acrlong{ad} provides a document with a simple explanation~\cite{smith1997scientist}. The only downside of the use of this is related with the microcontroller in use, but there are some other aspects to consider, so this will be explained next.

The microcontroller used was chosen due its low power capabilities, which end up being a very important aspect to consider in the future, for the development of the device. While developing, it started to be visible what can be a limitation of the microcontroller in use which is the execution time of \acrshort{fft}, an important aspect that can still be improved. 

Other aspect related to it is the precision and the resolution of the level identification. The number of samples considered in the \acrshort{fft} was 512, which gives, according to the obtained results, a interval of around 30 samples between the correspondent id of the empty bottle and the full bottle, which results in a coarse resolution. The algorithm can be set to use more samples, thus increasing the resolution, but that would require more \acrshort{ram} space. The reason for not using more samples was due the fact that the available~\acrshort{ram} is 4kB, which is enough to 2048 samples of 16bits, but there are other variables in~\acrshort{ram}, that limit the amount of samples below that value, which is not ideal because the algorithm perform better with more samples. This would not be considered a constrain if it was possible to use the free space available in the \acrshort{fram}, but the fact that the microcontroller in use does not have \acrshort{dma}(\acrlong{dma}), this would also increase the execution time in sampling process and the \acrshort{fft}. Besides, to store data in the \acrshort{fram}, is advised to disable interruptions, so it would not be possible to sample and store at the same time. The solution for this is the use of a microcontroller with more \acrshort{ram}. 

The last consideration is that all the samples collected along the development of the project are made available at the repository. The repository includes a document that contains information about each one of the experiments to allow in the future to perform tests without having to build any physical hardware.[Add the attachment and the link?]


%\addcontentsline{toc}{section}{References}
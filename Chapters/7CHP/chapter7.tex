\cleardoublepage
\chapter{Conclusions \& Future Work}\label{chap:finalremarks}
% To do: 
% Color Scheme
% Red - Not started
% Yellow - In progress
% Blue - Done, under approval
% Green - Done and approved
\todo[inline,color=red!40]{*1 - Conclusions}\label{sec:Conclusions}
\todo[inline,color=red!40]{*2 - Future Work}\label{sec:FutureWork}
\section{Conclusions}
This work started with the purpose to develop a device, capable of measuring the amount of liquid gas inside a \acrshort{lpg} bottle, without using an evasive technique and that would allow to use it in the bottles that  are already available in the market. Beside the device to this purpose, the final solution should also be able to transmit to the distributor the measured value, so he would be able to properly plan a distribution route, that would imply to develop a device, that would have \acrshort{iot} solution implemented with that purpose. 

Considering the practical objective of this work, is necessary to perform a background study, in order to analyze what type of work was already developed with similar purpose, before propose a architecture to the system. taking into account that the main focus is in the development of the device to measure the liquid level, the background study, was oriented to the research of technique to perform the measure of the liquid level and thus selecting the proper components to develop the work.

The starting point, was to find information of how to acquired data relative to the liquid level, i.e. what type of techniques can be used with that purpose, a couple were presented, each one with of them returning different types of information. Some of the mentioned were measuring the weight, the vibration or using ultrasound, in the end the choice fell to the measure the vibrations. 

The measure of the vibrations requires, the use of a sensor capable of measuring the small vibration produced, when stimulation the system. For that were selected two sensors capable of measuring the vibrations. From the same background study was found that there is a direct relation between the liquid level and the frequency of the vibration produced, to obtain the frequency of the vibration was necessary to process the data acquired from the sensors in a \acrshort{fft}. 

Taking into consideration the requisites of the system was proposed an architecture for the system, to acquire, process and return the liquid level. This proposal considers only a potential final product, since there was no idea of the type of results, the architecture itself was divided in several smaller steps, with the purpose of create the foundations for in the future build the proposed architecture, without the need to repeat this steps. 

The steps consist in a initial study of the response of the system to a stimulation, to verify if the approach is feasible and if there is in fact a relation between the liquid level and the frequency of the vibration. Followed by the test of the implementation of the \acrshort{fft} in a smaller and less powerful device. After it, testing the sensors in order to find if they are suited for the application and to determine of the same relation level vs frequency is found. And finally the development of an identification method of the liquid level, based on the conjugation of the last two steps. If there are positive results from all this steps, a setup can be develop according to the proposed architecture.

The tests performed in the different stages of the practical part of this work, revealed different distinct results. The first two steps, although with different purposes, where the ones that revealed the best results. The first, that was intended to study the response of the system to a stimulation, resulting in the confirmation of the first hypothesis, that there is a relation between the liquid level and the frequency of the vibration, was proved that the lower the liquid level the higher is the frequency produced in the vibration and the inverse for the maximum liquid level in the bottle, this test was also important in the determination of the variables to considered in the analysis in frequency, that were used in the tests that followed, mainly in the last two steps.

The second test, with his purpose to evaluate the accuracy of the implementation of the \acrshort{fft}, result in really precise results for the signals randomly generated, this guaranteed that the processing of the signals captured with the sensors and processed in the microcontroller, would also return viable data to determine the liquid level.

The third test, that focus on the study of the resulting data from the sensors, was the test that revealed the most disappointing results. Although there were positive results in the test, that allowed the execution of the last step, these test had different variables to considered and to control, with some of them without a proper manner to bypass them. The variable that was impossible to control, in a potential application, was the stimulation, it wasn't possible to obtain any results by stimulation the system with the solenoid, the only viable results were obtained by the stimulation with the hammer, that isn't consistent enough nor practical, but was enough to the execution of the last step. Another variable that was difficult to control was related to the coupling of the sensors, by using the load strap and the magnet to mount the accelerometer, we were able to obtain acceptable results, for the mentioned tests, the other options didn't reveal any viable results, this was the most difficult variable to control, even though there were some documents related to this, none of them had a similar application, the mechanical coupling of sensors to the system which is an unknow field of studies. 

The last step purpose was to evaluate if was possible, considering the the results of the processed data, to develop an algorithm capable of determine the liquid level. In this test, the results obtained for the identification algorithm weren't as good as expected, but was possible to determine it considering the first frequency peaks identified, as mentioned in the results analysis, it may be possible to obtain similar results with some changes. although there were acceptable results, this is also a step to improve.

In a final overview of all the work developed and the results obtained, there was several aspects that influenced the results, most of them already explored, there were another aspects that influenced, not directly in the results, but in the development itself. Considering these factors is important to mention that, even though the final purpose of the work wasn't complete, due the several constrains, was possible to obtain positive results. This results can be used in a future work as a good starting point to the continuation in the development of the final solution, they reveled what can be explored and what can be discard in a future. From the developed work and the problems faced along it, emerged various changes and suggestions that will be explored in the following section. 
\section{Future Work}
The development work in the context of the dissertation, although allowed the formation of a foundation to the development of a device with the capabilities of measuring the liquid level, as intended, still has several aspects that can be improved and other that need to have a different approach to archive the final goal of this work. In order to improve the already developed work and having in mind all the problems faced along the development of it, will be presented some suggestions in order to improved the work developed until this point. 

% Stimulation method
The first suggestion is concerning the stimulation of the system and in this case is actually two suggestions. If in the future the path to follow, is similar to what has been developed up to this moment, one suggestion is to find better method to stimulate the system, from the results is visible that hitting with a hammer return results, but isn't practical, so in this case the suggestion would be to use also use a solenoid, that would be capable to create a impact as strong as with an hammer. The other suggestion to the stimulation of the system is completely different to the one used, the approach could be with only the use of piezoelectric sensors, emitting a sine wave with a certain frequency to stimulate the capture the response with another piezoelectric, by varying the frequency and correlating is expected to determine the liquid level, this method was explored in chapter ~\ref{chap:stArt} in the work from ~\citeauthor{jahnLevelSensorFluids2014a}\cite{jahnLevelSensorFluids2014a}, as mentioned, is a completely different approach, but in their study proved to have good results.   
% Coupling 
One on the problems faced along the work was with the coupling of the sensors to the bottle, in the end there was only two methods that gave some results, which was with the use of a load strap strap and the magnet, the first ends up not being very practical to a future used, but when using the magnet is actually something that is practical to use and mount to the bottle, although the results weren't as good as with the load strap, this actual can be improved if a stronger neodymium magnet is used. Considering this and without any theoretical foundation, if the same was implemented with the magnet, it could also be possible to obtained similar results to what was obtained with the accelerometer.

In the hardware and the software developed there was always a problem related with the noise at the output of the sensors, for this case there is a suggestion that is widely used in signal processing for this type of situation and has a simple implementation is the use of Moving average filters, this type of filters are optimal to the reduction of random white noise, while keeping the the response. we won't get in details about this filters, for more information about the use of this filter~\acrlong{ad} provides a document with a simple explanation~\cite{smith1997scientist}. The only downside of the use of this if related with the microcontroller in use, but there is some other aspects to considered, so this will be explained next.

The microcontroller in used, was choose due his low power capabilities, which end up being a very important aspect to consider in the future, for the development of the device. While developing, started to be visible what can be a limitation of the microcontroller in use for instance, the execution time of \acrshort{fft} although was important has space to improve, since the longer it is working the higher will be the power consumption, for this case in specific it could be solved by increasing the the clock frequency used, for the tests was used 1MHz, the microcontroller can go up to 16MHz, this could reduce that time. 

Other aspect related to it is related with the precision and the resolution of the level identification, in the algorithm of identification, the number of samples considered in the \acrshort{fft} were 512, which gives, according to the obtained results, a interval of around 30 samples between the correspondent id of the empty bottle and the full bottle, which can give a higher error margin, the algorithm can be used to perform with more samples and thus increasing that interval, but that would require more \acrshort{ram} space. The reason for that is available is 4kB, which is necessary at least to 2048 samples of 16bits, but they aren't the only variable in the \acrshort{ram}, so limits the amount of samples below that value, which isn't ideal because the algorithm perform better and in this case, only performs with vectors that are a power of 2. But this isn't a constrain by itself, the data could be stored in the \acrshort{fram}, but the documentation of the device states that to access to it, all the interruptions shall be disable, if there wasn't for it it could be bypass, if the access time while storing, didn't exceed the the time between samples and could be done by \acrshort{dma}, which doesn't happen, there isn't \acrshort{dma} in this microcontroller.

But assuming that it wasn't the case and the samples were stored to the \acrshort{fram}, the access to read the value, process and store it may increase the processing time of the \acrshort{fft}. Combining all of this with the rest of the performed tasks, and if the the filter was added, would significally increase the processing time of the identification, which as mentioned will have costs related to the power consumption. The suggestion for this case is, either use a different microcontroller with higher \acrshort{ram}, at least the double, and \acrshort{dma} with the same low power capabilities, or use a more power microcontroller with much more \acrshort{ram}.

The last consideration, isn't a suggestion but a starting point for future work, attached to this document will be a link to drive folder, with all the samples collected along the development of it, inside the folder is a document that contains important information about each one of them, that will allow in the future to perform tests without having to build any physical hardware.[Add the attachment and the link?]


%\addcontentsline{toc}{section}{References}
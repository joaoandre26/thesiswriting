\cleardoublepage
\pagenumbering{arabic}
\chapter{Introduction}
\todo[inline,color=red!40]{*Change scope}
%%%%%%%%%%%%%%%%%%%%%%%%%%%%%%%%%%%%%%%%%%%%%%%%%%%%%%%%%%%%%%%%%%%%%%%%%%%%%%
\section{Background and Motivation} %% Change it latter
%% commands \gls, \Gls, \glspl and \Glspl used in glossary
%% command \footnote used in foot notes 

Energy plays an important role in our modern lifestyle and the development process of any country. Studies shown a strong correlation between the energy production/consumption and the economic/scientific development. Back in a couple of centuries ago, the primary source of energy was wood. As the industrialization evolution began, the need for energy sources with higher rates of energy efficiency emerged\cite{demirbasGlobalEnergySources2004}. The first successor was charcoal followed, a few years latter by oil and \acrshort{lpg}(\acrlong{lpg}). Compared with the other energy sources, \acrshort{lpg} burns very efficiently, emitting smaller amounts of pollutants gases. For example, to produce the same amount of energy produced by 1Kg of \acrshort{lpg}, burning it in a cookstove, it would be needed approximately at least 2.5Kg of charcoal and 21Kg of raw wood~\cite{File201403Multiple}. 

\acrshort{lpg} is the second biggest, non-renewable, source of energy in the world, with different consumption areas, as domestic, auto, industrial and agriculture. It has a highest consume in a domestic level, since is used to heating and cooking. The various types of \acrshort{lpg} are produced
by the extraction of natural gas, oil extraction and  oil refining. Propane, butane and natural gas are the most common types of LPG, used in house appliances~\cite{LiquefiedPetroleumGas}. More recently the energy sources have been replaced with renewable energy, to decrease the global warming and despite LPG huge impact, does not seem that it would not be replaced by other sources of energy, since recent studies developed what can be considered a renewable method of biosynthesis propane gas~\cite{kallioEngineeredPathwayBiosynthesis2014b}, commonly used in house appliances.   

The \acrshort{lpg} use in house appliances, usually is made via pluming systems in the cities, where the agglomeration of population is higher, which turns the pipelines gas installations more economic viable when compared with other regions. For smaller villages the solution for the \acrshort{lpg} distribution is based on the gas cylinders, which is divided in two main distribution systems, \acrlong{cccm}, where the final consumer is responsible for refueling the cylinder, most commonly used in cars, and \acrlong{bcrm}, where the final consumer only exchange a empty cylinder for a full one, being the supplier responsible for its refill, which require additional logistics for the suppliers distributions systems. For the second, it would be interesting for the suppliers and the cylinder retailers, to know the amount of gas in the cylinder of each final consumer of a certain region of supply.

This motivated a research in a accurate method of measuring the amount of gas in a \acrshort{lpg} cylinder and the information transmission to the supplier, taking advantage of the technology and using a \acrshort{iot}, as a way to transfer the information, suppliers would be able to efficiently plan their distribution routes. 

In a initial stage it will be explored different methods of measuring the level of gas in a \acrshort{lpg} cylinder and the the work developed is based in one of those methods. All the details will be explained along the document.

\section{Scope}
%\todo[inline,color=red!40]{Rewrite this, I don't like}
This work explores the possibility of develop a device capable of measuring the liquid gas inside a bottle in a non evasive way. 

Initially will be made a study of devices with similar purpose, this will allow the proposal of an architecture to a system capable of measure, process and identify the liquid gas inside the cylinder. This must be followed by the implementation of the proposed solution. In the end it will be performed individual tests to each piece of the system, concluding with the identification test. 
%%Older
%In order to develop a system that is robust/precise and featuring a system capable of communicate with the supplier and transmit the necessary information. A previous research must be made, to find similar work developed in the same field.

%Taking in consideration previous works, the identification of the measuring method must be made, looking to the pros and cons of all methods, and selecting the one that can provide the robustness and precise needed. For the communication system, should be based in IoT, and its range should cover most of the territory, to allow the information transfer of each LPG cylinder.

%The final device, should include all of the features mentioned, and take into consideration the components/development cost, making it relatively cheap to produce and within the business plan of the producer, and affordable for the final user.

\section{Outline}
This dissertation is divided in seven different chapters. The first chapter gives a general overview of the motivation behind the dissertation and defines the objectives to archive in the development of the work. 

The second starts by some background information, about previous works with similar purpose and a brief explanation of the theoretical components behind the work to develop. This chapter is the basis to define a solution to approach the problem presented.

The third is where is defined the approach to solve the presented work, it is presented all the necessary elements of the system to developed a general system architecture suggested.

The elements are defined in fourth and fifth chapters. The first relative to the physical, or hardware, elements that will be part of the system and the second is defined concerns about the software and the details of the integration of the hardware with the software.

The sixth chapter, is dedicated to present the different tests, their results and their analysis. The tests performed will serve as a basis to the continuation and integration of all parts in to what was proposed in chapter three. 

The work is finishes with Chapter seven, where are made some final considerations about the developed work, as well as suggestions for the future work.
\clearpage
%\printbibliography[heading=subbibliography]
%\addcontentsline{toc}{section}{References}
\cleardoublepage
\pagenumbering{arabic}
\chapter{Introduction}
\todo[inline,color=green!40]{*Missing OutLine}
%%%%%%%%%%%%%%%%%%%%%%%%%%%%%%%%%%%%%%%%%%%%%%%%%%%%%%%%%%%%%%%%%%%%%%%%%%%%%%
\section{Background and Motivation} %% Change it latter
\todo[inline,color=green!40]{*Edit reference number 2}
%% commands \gls, \Gls, \glspl and \Glspl used in glossary
%% command \footnote used in foot notes 

Energy plays an important role in our modern lifestyle and the development process of any country. And studies shown a strong correlation between the energy production/consumption and the economic/scientific development. Back in a couple of centuries ago, the primary source of energy was Wood, as the industrialization evolved the needed of energy sources with higher energy efficiency rates\cite{demirbasGlobalEnergySources2004}. The first successor was Charcoal followed, a few years latter by Oil and LPG. Compared with the other energy sources, LPG burns very efficiently, realizing smaller amounts of pollutants gases. For example to produce the same amount of energy produced by 1Kg of LPG, burning it in a cookstove, it would be need approximately at least 2.5Kg of Charcoal and 21.Kg of raw Wood~\cite{File201403Multiple}. 

LPG if the second biggest, non-renewable, source of energy in the world, with different consumption areas, as domestic, auto, industrial and agriculture. With the biggest consume in a domestic level, since is used to heating and cooking, with various types of LPG produced by the extraction of natural gas, oil extraction and oil refining, the most common used in house appliances are propane, butane and natural gas\cite{LiquefiedPetroleumGas}. More recently the energy sources have been replaced with renewable energy, to decrease the environmental impact and however LPG has bigger impact, doesn't seem that it won't be replaced by other sources of energy, since recent studies developed what can be considered a renewable method of biosynthesis propane gas~\cite{kallioEngineeredPathwayBiosynthesis2014b}, commonly used in house appliances.   

The LPG use in house appliances, usually is made via pluming systems in the cities, were the agglomeration of population is bigger, which turns the installation of gas pipelines more economic viable when compared with other regions. For smaller villages the solution for the LPG distribution is based on the gas cylinders, which is divided in two main distribution system, Consumer Controlled Cylinder Model (CCCM), where the final consumer is responsible for refueling the cylinder, most commonly used in cars, and Branded Cylinder Recirculation Model (BCRM), where the final consumer only exchange a empty cylinder for a full one, being the supplier responsible for its refill, which require additional logistics for the suppliers distributions systems. For the second, it would be interesting for the suppliers and the cylinder retailers, to know the amount of gas in the cylinder of each final consumer of a certain region of supply.

This motivated a research in a accurate method of measuring the amount of gas in a LPG cylinder and transmit that information to the supplier, taking advantage of the technology and using a IoT, as a way to transfer the information, suppliers would be able to efficiently plan their distribution routes. 

In a initial stage different methods of measuring the level of gas in a LPG cylinder will be explored, and the the work developed is based in one of those methods, all the details will be explained further in the document.

\section{Scope}
\todo[inline,color=red!40]{Rewrite this, I don't like}
In order to develop a system that is robust/precise and featuring a system capable of communicate with the supplier and transmit the necessary information. A previous research must be made, to find similar work developed in the same field.

Taking in consideration previous works, the identification of the measuring method must be made, looking to the pros and cons of all methods, and selecting the one that can provide the robustness and precise needed. For the communication system, should be based in IoT, and its range should cover most of the territory, to allow the information transfer of each LPG cylinder.

The final device, should include all of the features mentioned, and take into consideration the components/development cost, making it relatively cheap to produce and within the business plan of the producer, and affordable for the final user.

\section{Outline}

\clearpage
\printbibliography[heading=subbibliography]
\addcontentsline{toc}{section}{References}
\cleardoublepage
\chapter{Implementation} \label{chap:trans}
\section{Signal Simulations}
\section{FFT implementation}
\subsection{}
\section{Signal Acquisition} %Change latter
\section{Hardware Developed} %Change latter


% \begin{figure}[!htb]
%     \centering 
%         \begin{subfigure}[c]{\textwidth}
%             \centering
%             \input{Sections/3Transforms/Images/DFTSymmetry.tex}
%             \caption{}
%             \label{subfig:dft}
%         \end{subfigure}
%         \begin{subfigure}[c]{0.45\textwidth}
%             \centering
%             \input{Sections/3Transforms/Images/DCT1Symmetry.tex}
%             \caption{}
%             \label{subfig:dct1}
%         \end{subfigure}
%         \begin{subfigure}[c]{0.45\textwidth}
%             \centering
%             \input{Sections/3Transforms/Images/DCT2Symmetry.tex}
%             \caption{}
%             \label{subfig:dct2}
%         \end{subfigure}
%         \begin{subfigure}[c]{0.45\textwidth}
%             \centering
%             \input{Sections/3Transforms/Images/DCT3Symmetry.tex}
%             \caption{}
%             \label{subfig:dct3}
%         \end{subfigure}
%         \begin{subfigure}[c]{0.45\textwidth}
%             \centering
%             \input{Sections/3Transforms/Images/DCT4Symmetry.tex}
%             \caption{}
%             \label{subfig:dct4}
%         \end{subfigure}
%         \caption{Sequences generated in the first step of Table \ref{tab:DFTDCT}for the DFT and different DCTs. Filled dots correspond to the original sequence ((a) - \emph{DFT}; (b)) - \emph{DCT-I}; (c)) - \emph{DCT-II}; (d)) - \emph{DCT-III}; (e)) - \emph{DCT-IV}).}
%     \label{fig:2NSeq}
% \end{figure}
% \begin{lstlisting}
%     ./aomenc <INPUT-FILE> -h <HEIGHT> -w <WIDTH> -o <OUTPUT-FILE> --limit=10 -p 1 --cpu-used=8 --i420 --q-hist=64 --end-usage=q --cq-level=<CQ-LEVEL>
% \end{lstlisting}
\clearpage
%\printbibliography[heading=subbibliography]
%\addcontentsline{toc}{section}{References}